\documentclass[12pt,-letter paper]{article}
\usepackage{siunitx}
\usepackage{setspace}
\usepackage{gensymb}
\usepackage{xcolor}
\usepackage{caption}
%\usepackage{subcaption}
\doublespacing
\singlespacing
\usepackage[none]{hyphenat}
\usepackage{amssymb}
\usepackage{relsize}
\usepackage[cmex10]{amsmath}
\usepackage{mathtools}
\usepackage{amsmath}
\usepackage{commath}
\usepackage{amsthm}
\interdisplaylinepenalty=2500
%\savesymbol{iint}
\usepackage{txfonts}
%\restoresymbol{TXF}{iint}
\usepackage{wasysym}
\usepackage{amsthm}
\usepackage{mathrsfs}
\usepackage{txfonts}
\let\vec\mathbf{}
\usepackage{stfloats}
\usepackage{float}
\usepackage{cite}
\usepackage{cases}
\usepackage{subfig}
%\usepackage{xtab}
\usepackage{longtable}
\usepackage{multirow}
%\usepackage{algorithm}
\usepackage{amssymb}
%\usepackage{algpseudocode}
\usepackage{enumitem}
\usepackage{mathtools}
%\usepackage{eenrc}
%\usepackage[framemethod=tikz]{mdframed}
\usepackage{listings}
%\usepackage{listings}
\usepackage[latin1]{inputenc}
%%\usepackage{color}{   
%%\usepackage{lscape}
\usepackage{textcomp}
\usepackage{titling}
\usepackage{hyperref}
%\usepackage{fulbigskip}   
\usepackage{tikz}
\usepackage{graphicx}
\lstset{
  frame=single,
  breaklines=true
}
\let\vec\mathbf{}
\usepackage{enumitem}
\usepackage{graphicx}
\usepackage{siunitx}
\let\vec\mathbf{}
\usepackage{enumitem}
\usepackage{graphicx}
\usepackage{enumitem}
\usepackage{tfrupee}
\usepackage{amsmath}
\usepackage{amssymb}
\usepackage{mwe} % for blindtext and example-image-a in example
\usepackage{wrapfig}
\graphicspath{{figs/}}
\providecommand{\mydet}[1]{\ensuremath{\begin{vmatrix}#1\end{vmatrix}}}
\providecommand{\myvec}[1]{\ensuremath{\begin{bmatrix}#1\end{bmatrix}}}
\providecommand{\cbrak}[1]{\ensuremath{\left\{#1\right\}}}
\providecommand{\brak}[1]{\ensuremath{\left(#1\right)}}

\begin{document}
\begin{enumerate}

\item if one of the zeroes of the quadratic polynomial $x^2+3x+k$ is $2$,then the value of $k$ is
	\begin{enumerate}
			
		\item$10$ 
		\item$-10$ 
		\item$-7$  
		\item$-2$
	\end{enumerate}

\item the total number of factor of a prime numberis
	\begin{enumerate}
		\item$1$
		\item$0$
		\item$2$
		\item$3$	
	\end{enumerate}

			
\item the quadratic polynomial,the sum of whose zeros is $-5$ and their product is $6$,is\\
	\begin{enumerate}
		\item$x^2+5x+6$
		\item $x^2-5x+6$
		\item $x^2-5x-6$
		\item $-x^2+5x+6$
	\end{enumerate}		

\item the value of $k$ for which the system of equations $x+y-4=0a$ and $2x+ky=3$\\
	\begin{enumerate}
		\item$-2$
		\item$\neq 2$
		\item$3$
		\item$2$
	\end{enumerate}		
\item the HCF and the LCM of $12,21,15$ respectively are \\
	\begin{enumerate}
		\item$3,140$
		\item$12,420$
		\item$3,420$
		\item$420,3$
	\end{enumerate}		
\item the value of $x$ for which $2x,\brak{x+10}$ and $\brak{3x+2}$ are the three consecutive terms of an AP,is\\
	\begin{enumerate}
		\item$6$
		\item$-6$
		\item$18$
		\item$-18$
	\end{enumerate}		
\item the first term of an AP is $P$ and the commen difference is $q$,then its$10^{th}$ term is \\
	\begin{enumerate}
		\item$q+9p$
		\item$p-9q$
		\item$p+9q$
		\item$2p+9q$
	\end{enumerate}		
\item the distance between the points $\brak{a\cos\theta+b\sin\theta,0}$ and $\brak{0,a\sin\theta-b\cos\theta}$,is\\
	\begin{enumerate}
		\item$a^2+b^2$
		\item$a^2-b^2$
		\item$\sqrt{ a^2+b^2}$
		\item$\sqrt {a^2-b^2}$
	\end{enumerate}		
\item if the point $p$\brak{k,0} divides the line segment joining the points$\brak{2,-2}$ and B$\brak{-7,4}$ in the ratio 1:2, then the value of $k$ is\\
	\begin{enumerate}
		\item$1$
		\item$2$
		\item$-2$
		\item$-1$
	\end{enumerate}		
\item the value of $p$,for whichthe points a$\brak{3,1}$,B$\brak{5,p}$and c$\brak{7,-5}$ are collinear,is\\
	\begin{enumerate}
		\item$-2$
		\item$2$
		\item$-1$
		\item$1$
	\end{enumerate}		
\item in FIG.1,$\Delta ABC$ is circumscribing a circle, the length of BC is $1cm$.
	\begin{figure}[H]

		\includegraphics[width=\columnwidth]{./imagepavan1.png}
		\label{fig:fig1}
		\caption{triangle}

	\end{figure}
\item given \begin{align}
	\Delta ABC\Delta PQR,if\frac{AB}{PQ}=\frac{1}{3},then\frac {ar\brak{\Delta ABC}}{ar\brak{\Delta PQR}} 1cm.
	\end{align}
	
\item $ABC$ is an equilateral triangle of side $2a$, then length of one of its altitude is	
\item \begin{align}
	\frac{\cos80^\degree}{\sin10^\degree}+\cos59^\degree\csc31^\degree
\end{align}
\item the value of $\brak{\sin^{2}\theta+\frac{1}{1+\tan^{2}\theta}}$
\item the value of\begin{align}
\brak{1+\tan^{2}\theta}\brak{1-\sin\theta}\brak{1+\sin\theta} \end{align}
\item The ratio of the length of a vertical rod and the length of its shadow is $1$:$\sqrt{3}$ Find the angle of elevation of the sun at that moment?

\item Two cones have their heights in the ratio $1:3$ and radii in the ratio $3:1$. What is the ratio of their volumes?

\item A letter of English alphabet is chosen at random. What is the probabiliry that the chosen letter is a consonant.

\item A die is thrown once. What is the probability of getting a number less than $3$?
\item If the probability of winning a game is $0,07$, what is the probability of losing it?
\item If the mean of the first on natural number is $15$, then find $n$.
\end{enumerate}
\end{document}
